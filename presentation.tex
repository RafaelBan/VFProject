\documentclass[9pt]{beamer}
\usepackage[english]{babel}
\usetheme{default}
\usepackage{datetime}
\usepackage{adjustbox}
\usepackage{amsmath}
\usepackage{mathtools}
\usepackage{listings}
\usepackage{hyperref}
\usepackage{amsmath}
\usepackage{amssymb}
\usepackage{multicol}
\usepackage{multirow}
\usepackage{setspace}
\usepackage{algorithmic, algorithm}
\newcommand\tab[1][5mm]{\hspace*{#1}}

\newdate{date}{24}{01}{2024}

\title{\Large Verification of Binarized Neural Networks using alpha-beta-CROWN and Marabou}

\author{Rafael-Valentin Ban, Cosmin-\c Stefan Negureanu, Mihai-Iosif F\^{a}r\c tal\u a, Cristina-Larisa Petcu, M\u ad\u alina-Maria Radu}

\institute{West University of Timi\c soara\\Faculty of Mathematics and Informatics

Master Study Program: Software Engineering

\bigskip

Coordinator: Conf. Dr. M\u ad\u alina Era\c scu}

\date{\small{\displaydate{date}}
	
\bigskip

\begin{center}
\includegraphics[height=1.0cm]{FMI-03.png}
\end{center}
}

\AtBeginSubsection[]
{
  \begin{frame}[t]<beamer>{Overview}
    \tableofcontents[currentsection,currentsubsection]
  \end{frame}
}

\begin{document}
\begin{frame}[t]
  \titlepage
\end{frame}
\begin{frame}[t]{Overview}
  \tableofcontents
\end{frame}

\section{Introduction}
\begin{frame}[plain,c]{Introduction}
\setstretch{3} 
\begin{itemize}
    \item Motivation
    \begin{itemize}
        \item Improving verification rates of benchmark
    \end{itemize}
    \item Problem specification
    \begin{itemize}
        \item Self-driving
        \item Neural networks tool verifiers versus real life testing
    \end{itemize}
\end{itemize}
\end{frame}
\section{Dataset description}
\begin{frame}[plain,c]{Dataset description}
\begin{figure}[h]
\centering
\includegraphics[scale=0.8]{figure2.jpg}
\caption{Some images used in the German Traffic Signs Recognition Benchmark}
\end{figure}
\end{frame}
\begin{frame}[plain,c]{Dataset description}
\begin{figure}[h]
\centering
\includegraphics[scale=0.8]{figure3.png}
\caption{Properties file used for verification}
\end{figure}
\end{frame}

\section{Tools}
\begin{frame}[plain,c]{Tools}
\setstretch{3} 
\begin{itemize}
    \item alpha-beta-CROWN
    \begin{figure}[h]
    \centering
    \includegraphics[scale=0.17]{figure4.png}
    \caption{Rough explanation of efficient linear bound propagation}
    \end{figure}
\end{itemize}
\end{frame}
\begin{frame}[plain,c]{Tools}
\setstretch{3} 
\begin{itemize}
    \item Marabou
    \begin{itemize}
        \item based on SMT technology which answers questions about the properties of a neural network
        \item accepts multiple entry formats
        \item performs high-level reasoning on the network that can curtail the search space and improve performance
    \end{itemize}
\end{itemize}
\end{frame}
\begin{frame}[plain,c]{Tools}
\setstretch{3} 
\begin{itemize}
    \item Nnenum
    \begin{itemize}
        \item uses advanced abstractization for rapidly checking ReLU networks without sacrificing precisions
        \item written in Python
        \item utilizes GLPK for solving linear problems
        \item directly accepts ONNX files and VNNLIB property files
    \end{itemize}
\end{itemize}
\end{frame}
\section{Experimental Results}
\begin{frame}[plain,c]{Experimental Results}
\begin{center}
\begin{tabular}{ c c c c c c}
 \hline
 \textbf{\#} & \textbf{Tool} & \textbf{Verified} & \textbf{Falsified} & \textbf{Penalty}\\
 \hline
 1 & alpha-beta-CROWN & 0 & 39 & 3\\
 \hline
 2 & Marabou & - & - & -\\
 \hline
 3 & Nnenum & 0 & 0 & 46\\
 \hline
\end{tabular}
\end{center}
\end{frame}

\section{Conclusion}
\begin{frame}[plain,c]{Conclusion}
\setstretch{3} 
\begin{itemize}
    \item Posibility of verification improvement exists.
    \item Image verification is hard!
\end{itemize}
\end{frame}

\section{Demo}
\begin{frame}[plain,c]{Demo}
\setstretch{3} 
\begin{itemize}
    \item alpha-beta-CROWN\\
    \url{https://www.youtube.com/watch?v=cXHRKEpAh78}
    \item Marabou \& Nnenum\\
    \url{https://www.youtube.com/watch?v=YZIZdvPJcC8}
    \item Github link of the project\\
    \url{https://github.com/RafaelBan/VFProject}
\end{itemize}
\end{frame}
\end{document} 